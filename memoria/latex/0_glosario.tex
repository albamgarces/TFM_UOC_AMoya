%\pagenumbering{roman} 
%\setcounter{page}{1} 
\pagestyle{plain}

\chapter*{Glosario}

\begin{description}

\item[\textit{Acinetobacter baumannii}] Bacteria gram-negativa aeróbica. Puede formar parte de la flora epitelial y colonizar las vías respiratorias y digestivas y causar neumonía severa e infecciones del tracto urinario.
\item[Anotación del genoma] Identificación y asignación de funciones a los distintos elementos presentes en la secuencia genética de un organismo.
\item[Biocircos] Libreria gráfica de R para visualizar datos genómicos.
\item[BLAST] Basic Local Alignment Search Tool. Es una herramienta informática de alineamiento de secuencias. Es capaz de comparar una secuencia problema (\textit{query} ) con las secuencias que se encuentren en una base de datos.
\item[Cadena de procesos] Serie de procesos, generalmente lineales y unidireccionales, que toman unos datos de entrada y los transforma en datos de salida. El primer proceso toma los datos sin procesar como entrada, realiza una serie de acciones sobre ellos y envía los resultados al siguiente proceso. Termina con el resultado final producido por el último proceso de la línea. Por deficinición, cada uno de los procesos es autónomo, se pueden ejecutar fuera de la cadena.
\item[CDS] Coding Sequence o región de codificación de un gen.
\item[Duplicación genética] Duplicación de una región del genoma que engloba al menos un gen.
\item[Ensamblaje del genoma] Proceso mediante el cual se representan los cromosomas originales de la secuencia genómica de un organismo a partir de los múltiples fragmentos generados durante su secuenciación.
\item[\textit{Enterobacter}  spp.] Género de bacterias gram-negativas anaerobias facultativas. Muchas de las especies del género son patógenas y causa de infecciones oportunistas como el resto de integrantes del grupo ESKAPE. Pueden provocar infecciones en el tracto urinario y respiratorio.
\item[ESKAPE] Acrónimo que engloba a las especies bacterianas patógenas \textit{Enterococcus spp, Staphylococcus aureus, Klebsiella pneumoniae, Acinetobacter baumanni, Pseudomonas aeruginosa} y \textit{Enterobacter} spp. , que se caracterizan por ser especialmente virulentas al presentar diversos genes de resistencia a antimicrobianos.
\item[e-valor] Valor de significancia de los alineamientos obtenidos por BLAST teniendo en cuenta la probabilidad de que obtengan la misma puntuación por azar 
\item[FASTA] formato de fichero basado en texto, utilizado para representar secuencias. Comienza con una descripción en una única línea comenzada por el símbolo $>$, seguida por líneas de datos de secuencia. La simplicidad del formato lo hace muy sencillo de manipular y analizar.
\item[Flujo de trabajo] Conjunto de procesos que filtran o transforman datos. Puede ramificarse o ser lineal. Generalmente no hay un “primer” proceso claramente definido: los datos pueden ingresar al flujo de trabajo en múltiples puntos. Cualquier proceso podría tomar datos sin procesor como entrada y enviar sus resultados a otro proceso o generar un “resultado final”.
\item[GenBank] Base de datos de secuencias genéticas del NIH (National Institutes of Health de Estados Unidos) de disponibilidad pública. 
\item[Genes de resistencia] Expresión genética que permite a una bacteria sobrevivir en un ambiente con presencia de un antibiótico específico.
\item[GFF] General Feature Format. Formato para la descripción de los componentes de secuencias genómicas. Delimitados por tabulaciones con nueve campos por línea.
\item[Hit] Cada una de las secuencias similares obtenidas al realizar el alineamiento de secuencias a estudio con secuencias de referencia. 
Homología Situación en la que dos o más secuencias presentan un alto grado de similitud por lo que se deduce una relación ancestral común.
\item[\textit{Klebsiella pneumoniae}] Bacteria gram-negativa anaerobia facultativa ampliamente distribuida en el ambiente. Puede colonizar las vías nasofaríngeas y el tracto gastrointestinal provocando neumonías e infecciones urinarias. Es una especie muy frecuente en entornos hospitalarios que puede provocar infecciones graves en neonatos o pacientes de postoperatorio.
mutación adaptativa mutaciones que aumentan el éxito evolutivo y se transmiten de un organismo a otro perdurando en el tiempo.
\item[NCBI] National Center of Biotechnology Information. Es es parte de la Biblioteca Nacional de Medicina de Estados Unidos y almacena y actualiza constantemente la información referente a secuencias genómicas. También ofrece herramientas bioinformáticas para el análisis genómico a diferentes niveles y un índice de los artículos biomédicos en investigación. Estas bases de datos están disponibles en linea de forma gratuita.
patrones de expresión forma de expresarse característica de un conjunto se secuencias, indicando qué posiciones son más importantes y cuales pueden modificarse y cómo. Determinar patrones de expresión de proteínas es clave para determinar su función o estructura.
\item[Pipeline] ver cadena de procesos
\item[Plásmidos] Moléculas circulares de material genético extracromosómico, en bacterias y levaduras, capaces de replicarse de manera autónoma y transmitirse entre organismos.
\item[Pseudogen] Copia de un gen ya conocido y de función distinta que pueden haber modificado su funcionalidad o haberla cambiado completamente debido a la falta de intrones y otras secuencias de ADN esenciales para su función. Aunque son genéticamente similares al gen funcional original, no se expresan y suelen presentar numerosas mutaciones 
\item[\textit{Pseudomonas aeruginosa}]  Patógeno oportunista perteneciente al grupo de las bacterias gram-negativas aeróbicas. Infecta los pulmones y vías respiratorias provocando neumonías de caracter grave. También pueden infectar las vías urinarias o tejidos.
\item[Python] Lenguaje  de  programación  desarrollado  como  proyecto  de  código  abierto. Es  un  lenguaje interpretado,  lo  que  significa  que  no  se  necesita  compilar  el  código  fuente  para  poder  ejecutarlo.
\item[R] Lenguaje de programación para computación estadística que permite desarrollar gráficos de alto nivel.
\item[Script] Término coloquial a la secuencia de comandos de un programa sencillo
\item[Transposón] Secuencia de material genético capaz de desplazarse a diferentes partes del genoma de una célula, pudiendo causar mutaciones en el proceso.
\item[Workflow] ver flujo de trabajo


\end{description}
