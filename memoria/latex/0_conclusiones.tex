\chapter*{Conclusiones}
\label{chapter:conclusiones}

% Este capítulo tiene que incluir:

% Una descripción de las conclusiones del trabajo: Qué lecciones se han aprendido del trabajo?.

% Una reflexión crítica sobre el logro de los objetivos planteados inicialmente: Hemos logrado todos los objetivos? Si la respuesta es negativa, por qué motivo? 

% Un análisis crítico del seguimiento de la planificación y metodología a lo largo del producto: Se ha seguido la planificación? La metodología prevista ha sido la adecuada? Ha habido que introducir cambios para garantizar el éxito del trabajo? Por qué? 

% Las líneas de trabajo futuro que no se han podido explorar en este trabajo y han quedado pendientes. 

En este proyecto se ha diseñado una línea de trabajo para la búsqueda de duplicidades genómicas a partir de ficheros de anotación proporcionados por el usuario. Para llevar a cabo el proceso, se ha desarrollado una serie de programas o módulos en python y R que automatizan el análisis de los datos. El resultado final del proceso de análisis es una tabla de anotación de las proteínas duplicadas halladas en el genoma a estudio y una representación gráfica de las mismas.

El trabajo ha supuesto un gran reto tanto en su diseño como en la depuración del código y puesta a punto de los programas debido al gran esfuerzo que supone aprender un lenguaje de programación en pocos meses y resolver los problemas y errores del sistema que se iban dando.

No obstante, el desarrollo de este TFM ha permitido adquirir una serie de competencias que se resumen a continuación:

\begin{itemize}
    \item Crear un plan de trabajo desde el inicio para automatizar la búsqueda y representación de duplicidades génicas. Se ha tenido que ir adaptando a los plazos previstos conforme han surgido imprevistos que ralentizaban el avance del trabajo. También se han tenido que tomar decisiones a la hora de establecer prioridades para la consecución de los objetivos o relajar la depuración de los códigos.
    \item Aprender a tratar la anotación genómica a partir de distintos formatos. 
    \item Desarrollo del código para las diferentes funciones que llevan a cabo cada fase del proceso. Cada una de ellas debía ser autónoma y ser capaz de generar resultados por sí mismas. 
    \item Diseño de una herramienta capaz de automatizar todo el proceso a partir de pequeñas funciones autónomas que debían ser relacionadas entre sí para trabajar como un todo. Desde el tratamiento de los datos iniciales hasta la representación de los resultados obtenidos. se ha tenido que considerar múltiples factores y variables que hicieran la herramienta final lo más versátil y universal posible. Esta fase ha sido especialmente complicada debido a que había que tener en cuenta diversos factores que podían provocar la pérdida de autonomía de las funciones individuales.
    \item Adaptar scripts de código preexistentes a las necesidades del TFM ajustándolos a las características específicas del proyecto.
    \item Escribir nuevo código y profundizar en la comprensión de las estructuras de los lenguajes de programación $python$ y $R$. Lejos de lo que le podría parecer a alguna persona ajena al desarrollo de código informático, cada lenguaje tiene sus propias reglas y estructuras y se debe aprender a diferenciarlos bien para poder pasar con soltura de uno a otro.
\end{itemize}

De manera paralela a estos aprendizajes derivados directamente del desarrollo de los objetivos del TFM, también se han obtenido nuevos conocimientos en el área del trabajo colaborativo y de control de versiones con la plataforma github o la redacción de un texto técnico como es una memoria de final de máster en formato \LaTeX{}. La suma de todos estos nuevos conocimientos adquiridos, serán de mucha utilidad en otros ámbitos.

Los objetivos incialmente planteados incluían la interpretación de los resultados obtenidos en las cepas seleccionadas, pero hubo que prescindir de esa fase y alterar la planificación del trabajo a desarrollar. Esto ha sido debido en gran medida a fallos en la estimación del tiempo requerido para el desarrollo de los primeros módulos del proceso. No se tuvo en cuenta el tiempo requerido para la familiarización con la metodología, el aprendizaje profundo de un lenguaje de programación y la búsqueda de información para resolver los errores que se generaban debido a un código poco estable. Tampoco se consideró inicialmente la posibilidad de que los datos con los que se iban probando los paquetes no presentaran siempre la misma estructura o tipo de información que los que se encuentran en las bases de datos o generen \textit{de novo} el usuario. Esto produjo una continua necesidad de depurar y adaptar el código a las nuevas variables que iban apareciendo.

Aún así, se ha obtenido una herramienta funcional capaz de tomar cualquier archivo en los formatos compatibles y extraer la información necesaria para llevar a cabo exitosamente la búsqueda de genes duplicados, relacionarlos entre sí y representar de manera atractiva toda esta información. Esta herramienta podrá ser útil para la comunidad científica y avanzar en la comprensión de los mecanismos bacterianos. Especialmente cuando se da el caso de que son las bacterias más interesantes a nivel clínico por su virulencia y resistencia a antibióticos las que presentan mayores tasas de duplicación genética. Hasta ahora, el estudio de duplicidades génicas requerían de la continua intervención del investigador para pasar de una fase a la siguiente, con esta herramienta se permitirá realizar todo el proceso de manera automática y generar resultados con los que trabajar.


Como principal línea de trabajo futuro, claramente habría que conectar el módulo de R con el programa en python y permitir que se pueda descargar directamente los archivos de anotación de las bases de datos para terminar de automatizar todo el trabajo. También habría que incluir el código necesario para poder trabajar a gran escala con grandes cantidades de anotaciones diferentes y depurar en profundidad todo el proceso para mejorar la herramienta desarrollada. 

Resultaría interesante complementar los trabajos previos en duplicidades génicas en bacterias del grupo ESKAPE y localizar e identificar genes duplicados con particularidades relevantes. La inclusión de un método de análisis de elementos genéticos móviles y de genes asociados a virulencia y resistencia antimicrobiana junto con la caracterización de posibles correlaciones y patrones de duplicación terminaría de completar los temas que se plantearon inicialmente a la hora de establecer los objetivos del TFM.

Finalmente, sería de gran utilidad divulgativa realizar una guía completa de uso del programa para favorecer la comprensión de su manejo y los resultados obtenidos.

