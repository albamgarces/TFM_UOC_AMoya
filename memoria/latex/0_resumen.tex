\pagenumbering{roman} 
\setcounter{page}{1} 
\pagestyle{plain}

%%%%%%%%%%%%%%%%
%%% CREDITOS %%%
%%%%%%%%%%%%%%%%
\chapter*{Créditos/Copyright}



%\vspace{1cm}
\vfill

\begin{figure}[h]
    \centering
	\includegraphics[scale=1]{images/license.png}
\end{figure}

Esta obra está sujeta a una licencia de Reconocimiento -  NoComercial - SinObraDerivada

\href{https://creativecommons.org/licenses/by-nc-nd/3.0/es/}{3.0 España de CreativeCommons}.

%%%%%%%%%%%%%
%%% FICHA %%%
%%%%%%%%%%%%%
\chapter*{Ficha del Trabajo Final}

\begin{table}[ht]
	\centering{}
	\renewcommand{\arraystretch}{2}
	\begin{tabular}{r | l}
		\hline
		\multirow{3}{*}{Título del trabajo:} & Desarrollo de herramientas bioinformáticas en Python \\
		& y R para el análisis de duplicaciones génicas en bacte- \\
		& rias y visualización de datos \\
		\hline
        Nombre del autor: & Alba Moya Garcés\\
		\hline
        Nombre del colaborador/a docente: & José F. Sánchez Herrero\\
		\hline
        Nombre del PRA: & Ferran Prados\\
		\hline
        Fecha de entrega (mm/aaaa): & 5 de enero de 2020\\
		\hline
        Titulación o programa: &  Máster de Bioinformática y Bioestadística\\
		\hline
        Área del Trabajo Final: & Biología molecular y genética\\
		\hline
        Idioma del trabajo: & Español\\
		\hline
        Palabras clave & microbiología, duplicación génica, grupo ESKAPE\\
		\hline
	\end{tabular}
\end{table}

%%%%%%%%%%%%%%%%%%%
%%% DEDICATORIA %%%
%%%%%%%%%%%%%%%%%%%
\chapter*{Dedicatoria/Cita}

Breves palabras de dedicatoria y/o una cita.

%%%%%%%%%%%%%%%%%%%
%%% Agradecimientos %%%
%%%%%%%%%%%%%%%%%%%
\chapter*{Agradecimientos}

Si se considera oportuno, mencionar a las personas, empresas o instituciones que hayan contribuido en la realización de este proyecto.

%%%%%%%%%%%%%%%%
%%% RESUMEN  %%%
%%%%%%%%%%%%%%%%
\chapter*{Resumen}
\addcontentsline{toc}{chapter}{Resumen}

\onehalfspacing

Texto con la síntesis del proyecto, esto es, un texto en el cual se explica de manera concisa la definición del proyecto/problema abordado, sus objetivos/métodos de resolución, y los resultados y conclusiones (no puede ser una lista, sino un texto continuo redactado de manera estructurada). Si es necesario poner una referencia en este texto, ésta será anotada a pie de la misma página. En este apartado se puede usar un lenguaje más literario y coloquial que para el resto del documento.

El Abstract se escribirá por duplicado. Una de las versiones tiene que ser \textbf{obligatoriamente en inglés}. La otra versión tiene que estar escrita en catalán o español. En caso de no escribir el resto del documento en inglés, será necesario escribir la segunda versión del Abstract en el idioma utilizado para el resto de la memoria. La palabra Abstract se cambiará por ``\textbf{Resum}'' o ``\textbf{Resumen}'' en la versión catalana y española, respectivamente. 

Extensión recomendada: 250 palabras máximo.

Como escribir un buen Abstract (en inglés):

\href{http://www.ece.cmu.edu/~koopman/essays/abstract.html}{http://www.ece.cmu.edu/~koopman/essays/abstract.html}

\vspace{1.5cm}

\textbf{Palabras clave}: microbiología, duplicación génica, grupo ESKAPE

%%%%%%%
\chapter*{Abstract}
\addcontentsline{toc}{chapter}{Abstract}

\onehalfspacing

Texto con la síntesis del proyecto, esto es, un texto en el cual se explica de manera concisa la definición del proyecto/problema abordado, sus objetivos/métodos de resolución, y los resultados y conclusiones (no puede ser una lista, sino un texto continuo redactado de manera estructurada). Si es necesario poner una referencia en este texto, ésta será anotada a pie de la misma página. En este apartado se puede usar un lenguaje más literario y coloquial que para el resto del documento.

El Abstract se escribirá por duplicado. Una de las versiones tiene que ser \textbf{obligatoriamente en inglés}. La otra versión tiene que estar escrita en catalán o español. En caso de no escribir el resto del documento en inglés, será necesario escribir la segunda versión del Abstract en el idioma utilizado para el resto de la memoria. La palabra Abstract se cambiará por ``\textbf{Resum}'' o ``\textbf{Resumen}'' en la versión catalana y española, respectivamente. 

Extensión recomendada: 250 palabras máximo.

Como escribir un buen Abstract (en inglés):

\href{http://www.ece.cmu.edu/~koopman/essays/abstract.html}{http://www.ece.cmu.edu/~koopman/essays/abstract.html}

\vspace{1.5cm}

\textbf{Keywords}: microbiology, gene duplication, ESKAPE pathogens